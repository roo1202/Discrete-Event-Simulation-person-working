\documentclass[10pt,a4paper,twocolumn]{article} 
\usepackage[utf8]{inputenc} 
\usepackage[T1]{fontenc}
\usepackage[spanish]{babel}
\usepackage{hyperref}
\usepackage{graphicx}
\usepackage{float} 
\usepackage{tikz}
\usepackage{subcaption} 
\usepackage[export]{adjustbox}
\usetikzlibrary{shapes, shadows}

% Configuración de los márgenes (opcional) 
\usepackage[top=2cm, bottom=3cm, left=1cm, right=1cm]{geometry}

\title{Productividad de un trabajador, una simulaci\'on de eventos discretos} \author{ 
        Roger Fuentes Rodr\'iguez \\ 
        Jackson Vera Pineda \\
        Kevin Manzano Rodr\'iguez \\
        \\
        Universidad de la Habana\\
        C312 
        } 

\date{\today}

\tikzstyle{abstractbox} = [ draw=black, fill=white, rectangle, inner sep=10pt, style=rounded corners, drop shadow={fill=black, opacity=1} ] 
\tikzstyle{abstracttitle} = [fill=white]

\newcommand{\boxabstract}[2][fill=white]{ 
    \begin{center} 
        \begin{tikzpicture} 
            \node [abstractbox, #1] (box) { 
                \begin{minipage}{0.80\linewidth} 
                    \footnotesize #2 
                \end{minipage}}; 
                \node [abstracttitle, right=10pt] at (box.north west) {Abstract}; 
            \end{tikzpicture} 
        \end{center} }

\begin{document}

\maketitle

\boxabstract{Las técnicas de Pomodoro, populares por prometer aumentar la productividad mediante descansos regulares, han sido objeto de debate. Nos proponemos evaluar su eficacia mediante simulaciones basadas en la teoría de colas y un modelo de cadenas de Markov con datos similares. Se dise\~neraron experimentos controlados para comparar la eficiencia de seguir un método de descanso estructurado, o tomar descansos aleatorios. Se demuestra que no hay diferencias significativas en la cantidad de tareas completadas entre ambos enfoques. En la simulaci\'on el tiempo de trabajo siguió una distribución exponencial, mientras que la cantidad de tareas completadas siguió una distribución de Poisson. Basándonos en nuestros hallazgos, recomendamos explorar diferentes estrategias de descanso que se adapten mejor a cada individuo, en lugar de adherirse rígidamente a políticas de descanso predefinidas.}

\section{Introducción} 
La eficiencia laboral es un factor crítico en cualquier organización, siendo la productividad del trabajador uno de los pilares fundamentales para alcanzar objetivos empresariales. En este contexto, la gestión de tiempo y recursos se convierten en una tarea esencial, donde los descansos y las interrupciones juegan un papel significativo en la determinación de la productividad diaria. Conscientes de la importancia de optimizar estos aspectos, nuestro proyecto se enfoca en simular el comportamiento de un trabajador mediante eventos discretos, buscando modelar de manera realista las tareas, descansos e interrupciones que ocurren durante un día laboral típico.
\\

Nuestra propuesta se centra en desarrollar un modelo que refleje con precisión las dinámicas de trabajo, permitiendo analizar cómo estos eventos afectan la productividad del trabajador. Con la simulación, pretendemos identificar patrones y tendencias que puedan ser aprovechados para mejorar la eficiencia laboral. Específicamente, nos interesan explorar estrategias que permitan minimizar los efectos negativos de las interrupciones y maximizar los per\'iodos de descanso de manera óptima, con el objetivo de incrementar la productividad general. 
\\

Además, nuestro estudio busca realizar comparaciones estadísticas entre diferentes escenarios de trabajo, evaluando cómo varían los resultados bajo distintas condiciones de tareas, descansos e interrupciones. Este análisis comparativo permite ofrecer recomendaciones prácticas para la implementación de políticas laborales que promuevan un ambiente de trabajo más productivo.

\section{Metodología} 
Se dise\~na una simulación discreta para modelar el comportamiento de un trabajador en un entorno laboral típico. La simulación se diseño utilizando software de simulación adecuado, eligiendo entre opciones como \texttt{Arena}, \texttt{AnyLogic} o \texttt{SimPy}, dependiendo de las características específicas requeridas para nuestro modelo.

\subsection{Modelado de Eventos Laborales}
Dentro de la simulación, se modelan tres tipos principales de eventos:

\begin{enumerate}
    \item Tareas: Representan las actividades productivas que realiza el trabajador, con variabilidad en duración y prioridad.
    \item Descansos: Reflejan los intervalos de tiempo destinados al descanso del trabajador, considerando la duración y la frecuencia de estos descansos. 
    \item Interrupciones: Modifican el flujo normal de tareas, representando distracciones o interrupciones externas.
\end{enumerate}

Cada evento es modelado con parámetros probabilísticos que reflejan la variabilidad real en un entorno laboral.

\subsection{An\'alisis Est\'adistico}

Una vez definida la simulación, procedemos a ejecutar múltiples instancias de la simulación bajo diferentes escenarios para recopilar datos suficientes para el análisis estadístico. Los escenarios var\'ian en términos de duración de las tareas, frecuencia de descansos e intensidad de interrupciones.

Utilizamos herramientas estadísticas para analizar los resultados, incluyendo medidas de centralidad (como la media y mediana), dispersión (varianza y desviación estándar) y distribuciones de probabilidad para los tiempos de tareas, descansos e interrupciones. También realizamos pruebas estadísticas para comparar la productividad entre los diferentes escenarios.    


\subsection{Recomendaciones Basadas en Resultados}

Finalmente, basándonos en los hallazgos de nuestro análisis estadístico, derivamos recomendaciones prácticas para mejorar la productividad laboral. Estas recomendaciones pueden incluir estrategias para gestionar mejor los descansos y minimizar las interrupciones, así como sugerencias para la implementación de políticas laborales que promuevan un ambiente de trabajo más eficiente.

\section{Detalles de la implementación}

\subsection{Herramientas y tecnolog\'ias utilizadas}

\begin{enumerate}
    \item Lenguaje: \texttt{Python}, debido a su amplio soporte para operaciones matemáticas y manipulación de arrays, así como su comunidad activa que ofrece múltiples bibliotecas útiles para la simulación y análisis de datos.
    \item \texttt{SimPy} para llevar a cabo simulaciones complejas. \texttt{NumPy}, para el manejo eficiente de matrices y operaciones matemáticas, esencial para la implementación de la cadena de Markov, entre otras tenemos \texttt{Random}, \texttt{Scipy.stats}, \texttt{Pandas}.
\end{enumerate}

\subsection{Descripción del Proceso de Implementación}

\subsubsection{Teor\'ia de colas}

\begin{enumerate}
    \item Creación de los diferentes generadores de variables aleatorias.
    \item Creaci\'on de la clase Person con sus atributos, procesos y m\'etodos: Se especifica c\'omo interactua por medio de los m\'etodos que ser\'an tratados como procesos, y a la persona como un recurso; por la libreria \texttt{Simpy}.
    \item Realizaci\'on de la simulaci\'on: Se genera un $Environment$(ambiente) de \texttt{SimPy}, luego se ejecuta la simulaci\'on.
    \item Recogida de datos.
    \item Se calcula  una aproximaci\'on de la cantidad de veces que se est\'a en cada estado para luego compararla con el otro modelo propuesto. 
\end{enumerate}

\subsubsection{Cadena de Markov}

\begin{enumerate}
    \item Recogida de datos producido por el modelo de teor\'ia de colas.
    \item Modelado de la Cadena de Markov: Se definen los estados del sistema (trabajando, descansando, interrumpido) y se establecen las probabilidades de transición entre estos estados basándose en los datos proporcionados.
    \item Creación de la Matriz de Transición: Se construye una matriz de transición $3 \times 3$ que representa las probabilidades de cambio de estado.
    \item Simulación de Pasos de Tiempo: Se implementa un bucle que simula el paso de tiempo, donde en cada iteración se genera un número aleatorio para determinar el próximo estado del sistema basándose en las probabilidades de transición.
    \item Registro de Resultados: Durante la simulación, se registra el estado del sistema en cada paso de tiempo. Al final de la simulación, se analizan estos resultados para obtener informaci\'on sobre el comportamiento del sistema.
\end{enumerate}


Se puede consultar la implementación realizada desde este repositorio de GitHub.
\href{https://github.com/roo1202/Discrete-Event-Simulation-person-working}{Repositorio de GitHub}

\section{Resultados y Experimentos} 

\subsection{Principales interrogantes a tratar}
\begin{enumerate}
    \item Comparar qu\'e estrategia de administración de tiempo es m\'as productiva:
    \begin{enumerate}
        \item $24-6$, $24$ minutos de trabajo y $6$ minutos de descanso.
        \item $12-3$, $12$ minutos de trabajo y $3$ minutos de descanso.
        \item Free(aleatorio)
    \end{enumerate}
    
    \item ¿Cuál es la distribución de la cantidad de tareas completadas en una ventana de tiempo de 480 minutos?
    \item ¿Cómo afecta la duración de los descansos a la cantidad de tareas completadas?
    \item ¿Cómo afectan las interrupciones a la cantidad de tareas completadas?
    \item ¿Cómo se relacionan el número de descansos y el número de tareas completadas? 
\end{enumerate}

\subsection{Primer acercamiento}


\begin{figure}[H] % ht indica que la figura puede ir "aquí" o en la "parte superior" 
     % Centra la imagen en la página 
    \includegraphics[width=9.3cm, height=5cm, left]{tareas_completada_y_tiempo_de_trabajo.png} % Inserta la imagen 
    \caption{Tareas completadas y Tiempo de trabajo, con o sin interrupciones} % Agrega un título a la imagen 
    \label{fig:mi_imagen5} % Etiqueta para referenciar la imagen en el texto 
\end{figure}



Tenemos dos tipos de resultados obtenidos, evidentes y no-evidentes. Entre los no-evidentes tenemos que no es importante la metodología seguida con respecto a los descansos, se puede ser libre con el horario; el tiempo de trabajo distribuye exponencial y la cantidad de tareas distribuye poisson. Mientras en los evidentes encontramos que el tiempo de trabajo y la cantidad de tareas son proporcionales, y la cantidad de tareas realizadas se ve afectadas si se simula alguien propenso a interrupciones, observar ~\ref{fig:mi_imagen5}.

Se llevaron a cabo simulaciones de $480$ minutos ($8$ horas) con distintas metodologías, descansos cada $24$, $12$, o una cantidad aleatoria de minutos siguiendo una distribuci\'on exponecial con par\'ametro $\lambda = 24$; la duración de los descanso respectivamente queda : $6$, $3$, o una  cantidad aleatoria de minutos dada por una distribuci\'on exponencial con par\'ametro $\lambda = 6$. Todas estas variaciones se simularon con o sin interrupciones. 

A partir de las simulaciones con el modelo de colas, se recopilan los datos para conformar el modelo de cadenas de Markov, con el cual se llevan a cabo comparaciones, para ello se utilizan funciones que arrojan la diferencia modular entre los valores que devuelven ambos modelos ($f(x) = |y_1 - y_2|$ siendo $x,y_1,y_2$ el tiempo de simulaci\'on, el valor que devuelve el modelo de colas, el modelo que devuelve el modelo de cadenas, respectivamente). 

Para el criterio de parada se usa qu\'e tanto se aleja el estimador muestral del la poblaci\'on, y esto se comprueba cuando $\frac{\sigma}{\sqrt{k}} < c$ donde $c$ es el error que se est\'a dispuesto a cometer (la mitad de la longitud de confianza) con $95\%$ de confianza y $k$ el n\'umero de simulaciones. 

\subsection{El efecto de las interrupciones}

Es intuitivo pensar que las interrupciones pueden provocar una disminuci\'on en el desempeño de una persona. Aunque en nuestra simulación no se tengan en cuenta los efectos psicológicos de dichas perturbaciones, estas toman tiempo útil, por tanto podríamos preguntarnos \texttt{¿Cómo afectan las interrupciones a la cantidad de tareas completadas?}

Para responder esta interrogante hemos realizado las siguientes pruebas:
\begin{enumerate}
    \item Prueba Kolmogorov-Smirnov: test no paramétrico de Bondad de Ajuste para verificar si las muestras provienen de poblaciones significativamente diferentes con respecto al tiempo trabajado.
    \item Prueba Chi-Square se comprueba que la cantidad de tareas realizadas en ambas muestras es significativamente diferente.
    \item Prueba de Mann-Whitney U (Wilcoxon Rank): prueba no paramétrica sobre las medianas de ambas muestras con hipótesis alternativa "menor que"  (par\'ametro utilizado $alternative = 'less'$) para verificar cu\'al de las dos es la menor. Este test es usado (y no T-test por ejemplo) dando que no podemos asumir que la variable en cuestión está normalmente distribuida, ni que las condiciones que garantizan el $TCL$ (Teorema Central del L\'imite) se cumplen.
\end{enumerate}


En la prueba Kolmogorov-Smirnov dado que \'el $p$-$value < \alpha = 0.05$ existe evidencia suficiente para rechazar la hipótesis nula, por tanto consideraremos que las dos muestras (el tiempo de trabajo con y sin interrupción) no siguen la misma distribución.

En la prueba Chi-Square dado que \'el $p$-$value < \alpha = 0.05$ existe evidencia suficiente para rechazar la hipótesis nula, por tanto consideraremos que las dos muestras (la cantidad de tareas completadas y sin interrupción) no siguen la misma distribución.

En la prueba de Mann-Whitney U (Wilcoxon Rank) dado que el $p$-$value < \alpha = 0.05$ existe evidencia suficiente para rechazar la hipótesis nula, por tanto consideraremos la hipótesis alternativa: es significativamente mayor la cantidad de tareas completadas si no la persona no es interrumpida a si lo es.

\subsection{Distribuci\'on de variables de inter\'es}

Determinar la distribución que sigue un conjunto de datos en general no es una tarea sencilla y en muchos de los casos no es posible realizarse. Luego de observar los gráficos de los datos de la simulación, pueden notarse no muy evidentes, pero sí rasgos distintivos de algunas distribuciones conocidas, como exponencial en el caso del tiempo total trabajado, y poisson en el caso de la cantidad de tareas resueltas. 

Se comprueban dichas hipótesis:
\begin{enumerate}
    \item  Realizando prueba Kolmogorov-Smirnov con la variable de tiempo de trabajo, primero ajustando a los posibles valores exponenciales, y luego compar\'andola con datos de la distribución teórica.
    \item  Realizando prueba Chi-square con la cantidad de tareas completadas.
\end{enumerate}

En la prueba de Kolmogorov-Smirnov dado que \'el $p$-$value < \alpha = 0.05$ existe evidencia suficiente para rechazar la hipótesis nula, por tanto no es posible afirmar que el tiempo trabajado sin interrupciones sigue una distribución exponencial. Pero dado que el $p$-$value > \alpha = 0.05$ en el test Kolmogorov-Smirnov realizado al tiempo de trabajo con interrupciones, no existe evidencia suficiente para rechazar la hipótesis nula, por tanto es posible considerar que \'el tiempo de trabajo con interrupciones sigue una distribución exponencial seg\'un los datos arrojados por nuestra simulación. Más a\'un al realizarle el test a la muestra entera arroj\'o  un $p$-$value > \alpha = 0.05$ no existe evidencia suficiente para rechazar la hipótesis nula, por tanto es posible considerar que \'el tiempo de trabajo en general sigue una distribución exponencial seg\'un los datos arrojados por nuestra simulación.

En la prueba de Chi-Square dado que el $p$-$value > \alpha = 0.05$ no existe evidencia suficiente para rechazar la hipótesis nula, por tanto es posible considerar que la cantidad de tareas en general sigue una distribución Poisson seg\'un los datos arrojados por nuestra simulación.

\subsection{Mejor estrategia de administración}

Se propone verificar si alguna, y cu\'al de las estrategias "Pomodoro", tiene un resultado considerable en la cantidad de tareas resueltas de nuestra simulación. 

Para esta cuestión se realiz\'o:
\begin{enumerate}
    \item Crear gráficos de dispersión para visualizar el comportamiento del trabajo y el descanso, entre los grupos que siguen diferentes metodologías de planificación
    \item Prueba de Kruskal-Wallis no paramétrica para determinar si hay diferencias estadísticamente significativas entre medianas de los grupos de interés (es equivalente al ANOVA), dado que no podemos garantizar la normalidad de nuestros datos.
\end{enumerate}

\begin{figure}[H]
    \includegraphics[height=.80\linewidth, width=.97\linewidth]{tareas_completadas_y_descansos.png}
    \caption{Tareas Completadas y Descansos}
    \label{fig3a}
\end{figure}
\begin{figure}[H]
    \includegraphics[height=.80\linewidth, width=.97\linewidth]{Tiempo_de_trabajo_y_descansos.png}
    \caption{Tiempo de Trabajo y Descansos}
    \label{fig3b}
\end{figure}


En ~\ref{fig3a} se puede observar que ningún grupo sobresale más que otro en la cantidad de tareas completadas, puede notarse porque ningún color (grupos de inter\'es) se posiciona significativamente m\'as a la derecha que otro, en su lugar se posicionan aleatoriamente en el plano. Esto nos sugiere que puede no existir diferencia o mejora entre aplicar una estrategia de planificaci\'on u otra, ni siquiera analizando los grupos seg\'un las interrupciones (puntos y cruces).

Al hacer un analísis similar, se observa ligeramente un mejor desempeño en el tiempo de trabajo en estrategias con descansos más prolongados y más espaciados (ver figura ~\ref{fig3b} gráfica inferior derecha), pero no se asegura que sea significativo.


En la prueba de Kruskal-Wallis dado que el $p$-$value > \alpha = 0.05$ no existe evidencia suficiente para rechazar la hipótesis nula, por tanto se sustenta el análisis visual previamente realizado: La simulación arroja que NO existen diferencias significativas entre seguir alguna metodología (del tipo de las analizadas) o gestionar el tiempo aleatoriamente, respecto a la cantidad de tareas completadas**

\section{Modelo Matem\'atico}

El modelo esta conformado por un conjunto de estados y transiciones entre estos con ciertos valores, es decir un grafo dirigido ponderado. Se plantean los estados:  $$\{trabajando = W, descansando = B, interrumpido = I\}$$ La funci\'on de transici\'on: $$\delta(q_i) = \{(q_0, p_0), (q_1, p_1), (q_2, p_2)\}$$ donde $p_j$ representa la probabilidad de que estando en $q_i$ se vaya a $q_j$.

Se toma un conjunto de restricciones y supuestos de forma severa:
\begin{itemize}
    \item No se est\'a en dos estados a la vez
    \item cuando se entra a un estado no se sale hasta que se termina
    \item La determinación del siguiente estado no depende de información anterior.
    \item la suma de las probabilidades que salen de un estado suman $1$
\end{itemize} 

Para el c\'alculo de las probabilidades, se utiliza la probabilidad cl\'asica tomando en cuenta los datos que brind\'o el modelo basado en teor\'ia de colas. Se cuentan la cantidad de veces que se pasa de estar en el estado $q_i$ a estar en \'el estado $q_j$ denotemosla como $c_j$ entonces, $$p_j = \frac{c_j}{\sum_{i=1}^{3} c_i}$$.

De antemano es notable que las restricciones tomadas son muy fuertes por lo que los resultados no deber\'ian ser muy justos. Llevamos a cabo varias comparaciones, dada la representaci\'on como cadena de Markov, se cont\'o la cantidad de veces que se estaba en cada estado, para simulaciones tomando como entrada la cantidad de iteraciones; por su contrapuesto tenemos al modelo por colas y se quiere ver qu\'e tan pr\'oximo resulta la cadena a este, entonces tomamos como aproximaci\'on para cada par\'ametro a $\frac{Tiempo_{parametro}}{cantidad_{parametro}}$ o sea el tiempo que est\'a en los diferentes estados dividido por la cantidad de veces que se complete dicho estado.

Para todas las gr\'aficas se toma como las curvas rojas a la realizada por teor\'ia
de colas, y a la azul a la realizada por cadena de Markov.

\begin{figure}[H] % ht indica que la figura puede ir "aquí" o en la "parte superior" 
    \centering % Centra la imagen en la página 
    \includegraphics[width=5cm, height=5cm]{markov1.png} % Inserta la imagen 
    \caption{Par\'ametro$ = $tareas} % Agrega un título a la imagen 
    \label{fig:mi_imagen1} % Etiqueta para referenciar la imagen en el texto 
\end{figure}

\begin{figure}[H] % Otra figura centrada 
    \centering 
    \includegraphics[width=5cm, height=5cm]{markov2.png} 
    \caption{Par\'ametro$ = $descansos} 
    \label{fig:mi_imagen2} 
\end{figure}

\begin{figure}[H] % Y otra más 
    \centering 
    \includegraphics[width=5cm, height=5cm]{markov3.png} 
    \caption{Par\'ametro$ = $interrupciones} 
    \label{fig:mi_imagen3} 
\end{figure}

\subsubsection*{Comparaciones}
Notemos que aunque se ven diferentes guardan relaci\'on, pues siguen su crecimiento a lo largo del tiempo aunque con un alto grado de ruido en particular cuando se trata de las tareas ~\ref{fig:mi_imagen2}, pues recordemos que se tomo ciertas asunciones con respecto a estas, veamos como con las interrupciones~\ref{fig:mi_imagen3} si se tiene m\'as proximidad porque su comportamiento en ambos modelos es bastante similar 

\begin{thebibliography}{9} 
    \bibitem{ref1} Biwer, F., Wiradhany, W., oude Egbrink, M. G. A., \& de Bruin, A. B. H. 
    (2023).
    \textit{Understanding effort regulation: Comparing ‘Pomodoro’ breaks and self-regulated breaks.}. British Journal of Educational Psychology, 93(Suppl. 2), 353-367. \href{https://doi.org/10.1111/bjep.12593}{https://doi.org/10.1111/bjep.12593}


    \bibitem{ref2} L\'eon G. Faber, Natasha M. Maurits, Monicque M. Lorist. 
    \textit{Mental Fatigue Affects Visual Selective Attention}. Article  in  PLOS ONE · October 2012 
    \href{DOI:10.1371/journal.pone.0048073}{DOI:10.1371/journal.pone.0048073}

    \bibitem{ref1} Blaz Kos. 
    \textit{Interruptions at work Why are they a problem and the best ways to handle them}. 
    Spica.
    \href{https://www.spica.com/blog/interruptions-at-work}{https://www.spica.com/blog/interruptions-at-work}

    \bibitem{ref1} Hannah Ross. 
    \textit{The Impact of Interruptions on Productivity \& How to Combat Them}.
    Fellow.
    \href{https://fellow.app/blog/productivity/the-impact-of-interruptions-on-productivity-how-to-combat-them/}{https://fellow.app/blog/productivity/the-impact-of-interruptions-on-productivity-how-to-combat-them/}

\end{thebibliography}

\end{document} $